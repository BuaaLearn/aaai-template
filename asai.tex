\documentclass[letterpaper]{article}
\usepackage{hyperref}

\usepackage{tabularx}

%\usepackage{doublespace}
%\setstretch{1.2}

\usepackage{helvet}
\usepackage{times}
\usepackage{courier}
\usepackage{xspace}
\usepackage[T1]{fontenc}
\usepackage{mycv}
\usepackage[top=15mm,left=15mm,right=15mm]{geometry}
\begin{document}

\pagestyle{empty}

%Ueberschrift
\begin{center}
\huge{\textsc{Curriculum Vitae}}
\vspace{0.7\baselineskip}

\Large{\textsc{Masataro Asai}}
\vspace{0.5\baselineskip}

\large{
Ph.D Candidate

Department of General Systems Studies

Graduate School of Arts and Sciences, University of Tokyo
}

\vspace{0.8\baselineskip}

\normalsize{
 Gender: Male. Date of birth: March 28th, 1990. Present Citizenship: Japanese. \\
 Address: Saginuma-Viola 201 6-23-18 Arima Miyamae-ku Kawasaki Kanagawa, Japan. \\
 % Phone: +81-44-856-9009 \\
 Cell: +81-50-7576-3690. Email: guicho2.71828@gmail.com\ Skype: guicho2.71828\\
 Website: \url{http://guicho271828.github.io/}}
\end{center}


\section{Education}

\begin{CV}
 \item[04/2009--03/2013] Received B.Eng in Traffic Simulation at University of Tokyo, Japan.
 Specialization: Traffic Simulation, Multi-agent Model, Spatial Search.
 {\small Thesis: \emph{Distributed Cooperative Agents in Microscopic
 Traffic Simulation using St-RRT}; Advisor: S. Yoshimura,
 Professor. H. Fujii, Research Assistant Professor.}

 \item[04/2013--03/2015] M.A. in Artificial Intelligence at University of Tokyo, Japan.
 Specialization: Artificial Intelligence, Heuristic Search, Planning, Scheduling, Optimization
 {\small Thesis: \emph{Automated Cyclic Planning for Large Scale planning problems};
 Advisor: A. Fukunaga, Associate Professor}
\end{CV}

% \section{Undergraduate Thesis}
% 
% In the literature of large-scale traffic simulation,
% one of the major approaches is called a microscopic multiagent model.
% It simulates the behavior of each car (an agent)
% to produce the macroscopic emergent phenomena, e.g.\  traffic-jams,
% % The accuracy of the model are validated through
% % the comparison between the k-q (traffic density to average velocity)
% % curves of simulated and actual environments.
% % which can be heavily affected by the agent behavior.
% therefore, developping the appropriate agent model is the key
% factor to maintain the simulation accuracy.  One critical aspect of
% agent behavior is called a cooperative behavior. We have implemented an agent
% interaction model based on Spatiotemporal RRT(St-RRT) to
% simulate the cooperative behavior and evaluated the
% effectiveness of our approach.
% 
% \section{Masters Thesis}
% 
% In Automated Planning \& Scheduling (P\&S),
% domains such as factory assembly requires the planner program to
% assemble many identical instances of a particular product.  While modern classical
% planners can generate assembly plans for single instances of a complex
% product, generating plans to manufacture many instances of a product is
% beyond the capabilities of standard planners. We proposed ACP, a system
% which, given a model of a single instance of a product, automatically
% reformulates and solves the problem as a cyclic planning problem.  We
% showed that our ACP system can successfully generate
% cyclic plans for problems which are too large to be solved directly
% using standard planners.

\section{Work Experience}

\begin{CV}

\item[12/2011--09/2012] Internship at Metamoji.inc.  Prototyped a
  drawing-chat system for iPad. Both the server and the client side are
  written in Javascript with Node.js and Titanium Mobile.

\item[04/2012--08/2012] Teaching Assistant in ``Field Work -
 Introductory Course on Automobiles'', under
 Project Professor Kohei Kusaka.
 % The students learn the mechanism and the engineering
 % design decision of vehicles through vehicle maintenance experience.
 % prepared the maintainance materials such as lubricant, sealing and
 % oil filters, moved the vehicle which will be used for maintainance, and
 Advised the students from a safety standpoint while they
 learn the mechanism of a vehicle through the maintenance.
 
\item[04/2013--08/2013] Teaching Assistant in ``Experiments in
 Information and Environmental Sciences'', under
 Assoc.\ Prof.\ Haruo Saito.
 % Students are required to
 % conduct a specific experiment each week. Experiments include reading
 % the data from the various sensors and importing it to a computer,
 % during which the students are required to make use of digital circuits.
 % Tasks also include measuring the output of comparators, invert
 % amplifiers and filters that must be built by the students.
 Assisted students assemble and calibrate analog or digital
 circuits to read the physical value of the
 experimental equipment.

\item[03/2014--09/2014] Internship at LogicVein.inc,
 % LogicVein is
 a developer of a Configuration Management System
 for network routers and switches.
 % , an award-winning software that allows the users to
 % efficently manage the complex network configurations of remote routers and switches.
 % Useful batch orders can be triggered by their intuitive graphical interface.
 Worked on % . The materials
 % consist of: advertisement leaflets, blog articles and
 the technical product manual (more than 200 pages long) %  The materials
 % contain both the Japanese and English documents. Some staff are
 % native English speakers born in United States, who checked and highly
 % evaluated the quality of the English translation.
 and converted it into 
 % i also converted the Microsoft Word documents into
 % plain-text documents to set up a publishing system.  It compiles the
 % document into a \LaTeX{}-based pdf and multiple
 web pages, which is now bundled with their software.
\end{CV}

\renewcommand{\refname}{Publications}

\let\uline\relax
\nocite{Asai2014}
\nocite{Asai2014b}
\nocite{Asai2015}

\bibliographystyle{plain}
\bibliography{asai-references}

\section{Professional Services}

Subreviewer for Association for Advancement of Artificial Intelligence (2015).

One of my planning domain CELL-ASSEMBLY is added to SIGAPS ``Real and
Realistic Planning Domains'' by Patrik Haslum at
\url{http://users.cecs.anu.edu.au/~patrik/sigaps/index.php?n=Main.RealDomains}
.

\section{Technical Skills}

\begin{CV}
 \item[Programming Skills:] Logic-based, Constraint, Object-Oriented,
 Functional, Type-based programming. Test-driven development, Metaprogramming, DSL, Compile-time optimization.
 \item[Programming Languages:] C, C++, Java, Python, Ruby, Bash,
 Javascript/Coffeescript, Emacs Lisp, Common Lisp
 \item[Software skills:] Git, jQuery, Titanium Mobile,
 Node.js, Socket.IO, bootstrap.js, websocket, CSS3, HTML5, Wordpress, NIS, NFS,
 Torque, network monitoring, Unix-based OSes
 \item[Hardware skills:] Digital and Analog circuits, microcontrollers(Arduino,PIC), machining/welding
\end{CV}

\section{Language Ability}

\begin{CV}
 \item[Japanese:] native
 \item[English:] TOEFL 105/120 (Reading:29/30, Listening:29/30,
 Speaking:22/30, Writing:25/30, Dec 2014). Enjoys the daily discussion
 on programming on Reddit, Github, Skype and IRC channel.
\end{CV}

\section{Community Services / Other activities}

(2012--present) \href{https://github.com/guicho271828}{Open source activities on Github}.

(2013--present) Managing Torque-based compute clusters in the Laboratory. Maintaining
the NFS/NIS-based file sharing and login name synchronization.

(2011--2012) Professional engineering activity on engine modification on Mazda Miata
'89-'04 under Project Professor Kohei Kusaka. Activities include full
engine rebuilding, mechanical engineering (compression ratio
optimization), fuel map / ignition timing map optimization, development
of variable resonance intake controller.

(2011) Certification in ``basic course on machining technique'' by Prof. Ryu Chikayama.

(2005--2007) Development of Bipedal robot with embedded microcontroller
(Microchip\textregistered PIC and analog servo motors)

\end{document}
