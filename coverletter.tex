%%%%%%%%%%%%%%%%%%%%%%%%%%%%%%%%%%%%%%%%%
% Short Stylish Cover Letter
% LaTeX Template
% Version 1.0 (28/5/13)
%
% This template has been downloaded from:
% http://www.LaTeXTemplates.com
%
% Original author:
% Stefano (http://stefano.italians.nl/archives/63)
%
% License:
% CC BY-NC-SA 3.0 (http://creativecommons.org/licenses/by-nc-sa/3.0/)
%
%%%%%%%%%%%%%%%%%%%%%%%%%%%%%%%%%%%%%%%%%

\documentclass[12pt]{article}

\usepackage{marvosym}
\usepackage{helvet}
\usepackage{times}
\usepackage{courier}
\usepackage{xspace}
\usepackage{hyperref}
\usepackage[T1]{fontenc}
% \usepackage{mycv}
\usepackage[top=27mm,left=27mm,right=27mm,bottom=27mm]{geometry}

\pagestyle{empty}

\begin{document}

{
\large
Response to the selection criteria for the Post-doc position (Job ID 516745)\\
}

Masataro Asai

% Dear Dr. Patrik Haslum,
% 
% \vspace{1em}

\setlength{\parskip}{0.3em}
% PARAGRAPH ONE: State the reason for the letter, name the position or
 % type of work you are applying for and identify the source from which
 % you learned of the opening.\\ 

% PARAGRAPH TWO: Indicate why you are interested in the position, the
% company, its products, services - above all, stress what you can do
% for the employer. If you are a recent graduate, explain how your
% academic background makes you a qualified candidate for the
% position. If you have practical work experience, point out specific
% achievements or unique qualifications. Try not to repeat the same
% information the reader will find in the resume. The purpose of this
% section is to strengthen your resume by providing details which bring
% your experiences to life.\\ 


% Please find enclosed CV in the application.

\textbf{1. A PhD in computer science, or a related field, with a focus
relevant to the project, and a track record of independent research in
this area evidenced by publications in peer-reviewed journals and
conferences, a record of developing and maintaining collaborations, and
by other measures such as awards, invitations to give talks at leading
conferences, etc.}

I am going to receive a Ph.D degree from the current institution on March 2018.
I have strong track record in peer-reviewed conferences/journals on planning and scheduling,
including AAAI, IJCAI, JAIR, ICAPS.

I can write efficient, maintainable and readable programs,
favoring collaboration, code review, good coding standards and other miscellaneous programmer's virtue,
as evidenced in my open source activities on github:
1) Most projects are tested under Continuous Integration services for reliability.
 % I occasionally accept bug fixes, merged without breaking the code.
2) I regularly write concise documentation, using well-defined terms and ideas.
% 4) With an ability to fully leverage the metaprogramming, I can write even the most
%  complicated kinds of compile-time optimization within 500 lines of code.
% I can use these abilities for the rapid prototyping and the fast development cycles which ensure
%  the success of the entire project.
3) I have projects of more than 50 stars which occasionally gets some requests and questions from the users.
 I help my users understand and use my product, through which I improve the documentation or the setup process.

I received an award from Japanese Society for Artificial Intelligence,
the largest local AI conference in Japan.
Although not being competitive/peer reviewed in nature,
this is a place for networking where most Japanese AI researchers attend,
and the award is a proof of excellence among Japanese researchers.

\vspace{1em}

\textbf{2. Significant expertise that is relevant to research in automated planning, specifically in planning under
(quantitative) uncertainty and partial observability, knowledge engineering for automated planning, with
documented ability to perform innovative research, implement algorithmic innovations and carry out
experimental and empirical studies. Expertise in network security, pentesting and/or software/system
vulnerability analysis is also desirable.}

I worked on planning under uncertainty during the internship at IBM Research Ireland.
My latest publication in IJCAI is based on a hybrid system which uses
AO* for the low-level journey planning and A* for higher-level task scheduling
which invokes AO* multiple times. 
During the development, I significantly modified the AO* code, which was originally designed for one-off planning,
so that it integrates well to the higher-level planning code with additional caching between multiple invocations.
I also have familiarity with LAO* and more popular algorithms such as LRTA* or DP for MDP planning.

I also contributed to your SIGAPS project with a large-scale
manufacturing domain. My paper on ICAPS14 and ICAPS15 handles such
problems effectively through an automated Knowledge Engineering
techniques (or macro actions).

\vspace{1em}

\textbf{3. Ability to work as part of a team and towards successful
project delivery and deadlines, and to coordinate and take
responsibility for some team activities.}

During the internship at IBM Research, I was involved in a team
consisting of planning researchers and the domain experts.
My internship period was significantly reduced from the standard length
due to the restriction from my own funding, which set a challenging deadline
for implementing the program and conducting the preliminary experiments.

\textbf{4. Excellent oral and written English language skills and a
demonstrated ability to communicate and interact effectively with a
variety of staff and students in a cross disciplinary academic
environment and to foster respectful and productive working
relationships with staff, students and colleagues at all levels.}

English language skills should be manifested in this document alone, as
well as in my past publications, in my past presentations in the conferences,
and in my personal communication with Dr. Haslum at the last ICAPS conference.

\textbf{5. Ability to supervise student research projects at the
undergraduate, honours, graduate coursework and masters or PhD level.}

I contributed to the several projects of a former masters graduate student
by developing the RESTful API for a solver as well as running the
experiments for his algorithm, in addition to giving the general advice.
The former was used in his graduate thesis,
and the latter became a paper in HSDIP workshop in 2016.
I also give a significant amount of advice to a foreign masters student in the laboratory
in the development of a planner based on Fast Downward.

\textbf{6. Ability and commitment to win bids for competitive external
funding to support individual and collaborative research activities.}

I received a government-led research funding which offers
stipends and independent research budget (1,000,000JPY annual) and will continue to collect external fundings.
The application goes through a standard process where the applicant writes 
a proposal and the process is highly selective (\href{https://www.jsps.go.jp/j-pd/pd_saiyo.html}{21.8\%}).

\textbf{7. Ability and willingness to teach at all levels.}

My teaching assistant role in 2013 involved a lecture using LEGO mindstorm NXT,
where a group of students not familiar with programming is tasked to code a simple
agent which avoids the obstacles.

Other than that, I do not have the official records of
teaching experience related to CS, for example, teaching the basic programming as part of a university lecture.
However, I do not find any problem with teaching, or even like to teach,
partly because long back in the highschool I taught C for the younger students in my extracurricular activities.


\textbf{8. A demonstrated high-level understanding of equal opportunity
principles and a commitment to the application of these policies in a
University context.}

I understand and am commited to follow the respect and inclusion policy at ANU, which consists of:

\begin{itemize}
 \item Gender Equity, including the use of gender inclusive language
       avoiding masculine nouns (invisibility), unnecessary emphasis on
       a specific gender (inferiority) or trivialization of either
       gender, as well as the recognition and respect for sexual
       minorities (LGBTIQ*).
 \item Respect for Indigenous Australian (Australian Aboriginal and Torres Strait Islander)
 \item Understanding and the respect for people with various type of
       disabilities, including but not limited to, visual, hearing or
       mental problems, or people with intellectual disability, brain
       injury, autism spectrum or physical disability.
 \item Policy against racial and age discrimination.
\end{itemize}

% Moreover, I am a fast learner. I wrote my first conference paper within 6 months after changing the
% major from Traffic Simulation to Heuristic Search (which I have no
% experience at that time), moving in to the different laboratory as a stranger.

% I am a fast and accurate writer, with a keen eye for detail and I should
% be very grateful for the opportunity to progress to market reporting. I
% am able to take on the responsibility of this position immediately, and
% have the enthusiasm and determination to ensure that I make a success
% of it.

% PARAGRAPH THREE: Request a personal interview and indicate your
 % flexibility as to the time and place. Repeat your phone number in the
 % letter. End the letter by thanking the employer for taking the time
 % to consider your credentials.\\ 

\end{document}