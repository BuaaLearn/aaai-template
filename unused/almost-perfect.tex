% NOTE: Let's not overemphasize the Helmert and Rogers results on almost perfect heuristics.
% Although we'll certainly cite helmert2008good, we should not make helmrt2008good central
% to the motivation of the paper

Since the advent of delete-relaxation and abstraction heuristics in
admissible Classical Planning, much of the interest was focused on improving
the accuracy of these heuristic functions to prune more nodes from the
search space.
% 
However, recent work by Helmert and Roger
\shortcite{helmert2008good} claims that, even with an optimistic
assumption of \emph{almost-perfect heuristics}, \astar still has an
exponentially large plateau. They conclude that the further performance
improvement requires the techniques orthogonal to the heuristic
function, such as symmetry breaking, domain reduction, factored planning
etc.










Trivially, the best possible admissible heuristic function is $h^*$ itself, which is
called \emph{perfect heuristics}.
It is known that with a perfect heuristics the planner do not have to
conduct any search: There are no multiple possibilities that the
planner should examine on each node.
However, computing $h^*$ is PSPACE-Complete,
which is as difficult as solving the problem itself and is not
practical.

\emph{Almost perfect} heuristics $h_c$ is a class of similar
impractical, theoretical functions which is also PSPACE-Complete to
compute \cite{helmert2008good}.  It has a constant error $c$ from the
perfect heuristic $h^*$, i.e., $h_c=h^*-c$.  The important finding by
\citeauthor{helmert2008good} is that, even if we assume this intractable and
optimistic heuristic function was obtained, the number of the nodes in the final
plateau of the search space becomes exponentially large as the problem size
increases. Based on this fact, they claimed that relying on the
improvement of the heuristic functions has a limitation,
and the researchers should seek the other, orthorgonal improvements.
