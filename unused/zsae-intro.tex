

ZSAE addresses three issues caused by the unstable propositional symbols of Latplan.
% The first two issues are directly related to the unstable symbols.
% Unstable symbols affect search algorithms in two ways:
First, they increase the number of node generations because it confuses the duplicate detection mechanism.
The second, and more important, issue is that they may make the state space graph disconnected because certain edges
are connected to only one random variant of the symbolic representation of real-world states.
This makes planning to fail without a solution or return a suboptimal solution.
% In fact, in the appendix section in the Arxiv version of the original paper \cite{Asai2018},
% the authors stated that they used \emph{state augmentation} technique
% which circumvent this problem by sampling states from the same image multiple times.
Thirdly, the system is sensitive to the hyperparameters, and with an inappropriate set of parameters,
the propositional symbols are highly unstable.
Specifically, if we set the number of propositions ($=$ number of nodes in the latent layer) too high, the network
has an excess capacity which results in too many unused, unstable propositions.







To address these problems,
we introduce an additional regularization penalty function for the neural network 
that makes the learned propositions more ``stable''. In the resulting neural network,
Zero-Suppressed State AutoEncoder (ZSAE, \refig{zsae-overview}), this penalty function
guides the network optimization so that unused propositions tend to 
take the value of zero (false) instead of random values,
resulting in a more stable representation.
The stability can be measured by taking the variance of the representation from the same set of inputs.
We also show that a more stable representation results in a higher success rate of classical planning.

Also, with this additional penalty, the number of bits required to represent the same function
is minimized and the network becomes less sensitive to the hyperparameters.




The idea behind ZSAE is similar to the Zero-Suppressed Binary Decision Diagram \cite{minato1993zero},
a variant of Binary Decision Diagram \cite{bryant1986graph} with an alternative node reduction rule:
Nodes whose 1-edge going to constant 0-node are pruned.
