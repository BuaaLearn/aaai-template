We can have several important observations from these results.  Firstly, in a plateau, \textbf{the heuristic functions are not used at all, nor the search is guided at all}. This observation holds even if we combine several nondominating heuristics e.g. \lmcut and M\&S, regardress of the method, e.g., taking the maximum, using portfolio or utilize them as the first tiebreaking strategy. It is still possible that a plateau is encountered, since it is not a perfect heuristics yet!
Such a plateau is known to be inevitable even if we have an almost perfect heuristics $h_c$, and it is impossible to improve upon $h_c$ --- if it could, the result would be a perfect heuristics or an inadmissible heuristics. Therefore, this problem cannot be solved by improving the heuristic accuracy, which is the currently dominating meta-strategy to improve the planner performance.

Secondly, there is no legitimate reason which supports each tiebreaking strategy.
As long as $f$-value is the first sort key, any tiebreaking criteria are admissible. \textbf{$h$ and FIFO are just heuristically chosen by the implementer of the planner.} Nor are there any reason to choose LIFO or Random tie breaking. Moreover, the different seed value of a Random tiebreaking yield the different search behavior and different result. (In all of our experiment we fixed the seed to 1.)

Based on these observation, the next step we have taken is to develop a new
portfolio-based multi-tiebreaking strategy \textbf{which is orthogonal to
the approach of improving the heuristic accuracy.}

The following sections are devoted to giving further analysis on the
reason behind their behavior.
