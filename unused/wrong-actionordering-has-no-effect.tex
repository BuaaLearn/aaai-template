
\subsubsection{Do Action Orderings Affect the Performance?}

Finally, we show that the performances of tie-breaking strategies are
not affected by the bias introduced in the action orderings in the PDDL
domain definition.
In \textbf{satisficing planning}, tie-breaking behavior
is known to be affected by the accidental bias in action ordering in the PDDL domain
definition \cite{vallati2015effective}.

We tested $[f,h,\lifo]$ and $[f,h,\fifo]$
on 3 individual sets of domains in which the
original names of action schema are uglyfied into random strings. This is because
the Fast Downward code base has a preprocessing
step which sorts the action schemas according to the dictionary
order, and it affects the order of the applicable actions of each search node.
% 
The coverage results are shown in \reftbl{actionordering-robustness}.
The result suggests that in cost-optimal planning, the effect of action
ordering has little effect on the performance.
\todo{maybe better to increase the number of runs.}
% We analysed these results using statistical methods.
% We first applied Bartlett test to test if the sample groups wrto each
% set of randomized domains share the same variance.
% There is no significant difference between the variances of the sample
% groups with confidence $p=0.00 < 0.05$.
% 
% Since the variances are the same, we applied one-way ANOVA to test if
% the sample groups have the same mean values.
% %%
% % Since the variances are not the same, we applied 
% % non-parametric Kruskal-Wallis test to test if
% % the sample groups have the same mean values.
% This was verified with a confidence $p=0.00<0.05$.

\begin{table}[tb]
 \centering \relsize{-3}
 \begin{tabular}{|c||c|c|c||c|c|c|}
\hline                  
 Domain & \rotatebox[origin=l]{90}{lmcut,rd,randomx,2280}   & \rotatebox[origin=l]{90}{lmcut,rd,randomx,2432}   & \rotatebox[origin=l]{90}{lmcut,rd,randomx,15314}   & \rotatebox[origin=l]{90}{lmcut,rd,randomx,2280}   & \rotatebox[origin=l]{90}{lmcut,rd,randomx,2432}   & \rotatebox[origin=l]{90}{lmcut,rd,randomx,15314}    \\
\hline                  
 sum(1104) &  571 &  573 &  571 &  53054630959 &  52855052463 &  53055118088 \\\hline
 sum(1104) &  571 &  573 &  571 &  52954806534 &  52954410880 &  53154246049 \\\hline
 sum(1104) &  571 &  573 &  571 &  52854208990 &  52654902163 &  53054545808 \\\hline
 sum(380) &  170 &  169 &  170 &  21044249397 &  21139379995 &  21045425987 \\\hline
 sum(380) &  170 &  169 &  170 &  20845700303 &  20744544273 &  20742668307 \\\hline
 sum(380) &  170 &  169 &  170 &  20941436904 &  20941330432 &  20857182304 \\\hline
 sum(260) &  128 &  126 &  127 &  12953782527 &  13146086022 &  13041215376 \\\hline
 sum(260) &  128 &  126 &  127 &  12555898212 &  13138649739 &  12267203886 \\\hline
 sum(260) &  128 &  126 &  127 &  12952354353 &  12761488656 &  12948386130 \\\hline
\end{tabular}

\begin{tabular}{|c||c|c|c||c|c|c|}
\hline                  
 Domain & \rotatebox[origin=l]{90}{lmcut,rd,randomx,2280}   & \rotatebox[origin=l]{90}{lmcut,rd,randomx,2432}   & \rotatebox[origin=l]{90}{lmcut,rd,randomx,15314}   & \rotatebox[origin=l]{90}{lmcut,rd,randomx,2280}   & \rotatebox[origin=l]{90}{lmcut,rd,randomx,2432}   & \rotatebox[origin=l]{90}{lmcut,rd,randomx,15314}    \\
\hline                  
 sum(1104) &  571 &  573 &  571 &  53054630959 &  52855052463 &  53055118088 \\\hline
 sum(1104) &  571 &  573 &  571 &  52954806534 &  52954410880 &  53154246049 \\\hline
 sum(1104) &  571 &  573 &  571 &  52854208990 &  52654902163 &  53054545808 \\\hline
 sum(380) &  170 &  169 &  170 &  21044249397 &  21139379995 &  21045425987 \\\hline
 sum(380) &  170 &  169 &  170 &  20845700303 &  20744544273 &  20742668307 \\\hline
 sum(380) &  170 &  169 &  170 &  20941436904 &  20941330432 &  20857182304 \\\hline
 sum(260) &  128 &  126 &  127 &  12953782527 &  13146086022 &  13041215376 \\\hline
 sum(260) &  128 &  126 &  127 &  12555898212 &  13138649739 &  12267203886 \\\hline
 sum(260) &  128 &  126 &  127 &  12952354353 &  12761488656 &  12948386130 \\\hline
\end{tabular}

 \caption{Results showing the total coverages and evaluations
 of $[f,h,\lifo]$ and $[f,h,\fifo]$
 on three uglified versions of the same set of domains. The effect
 of action ordering is too small to affect the coverage.}
 \label{actionordering-robustness}
\end{table}
