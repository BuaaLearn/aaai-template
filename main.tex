%%%%%%%%%%%%%%%%%%%%%%%%%%%%%%%%%%%%%%%%%%%%%%%%%%%%%%%%%%%%%%%%
\begin{hidden}
 
Like any other software, the key part of collaborative development is *documentation*. In this camera-ready version, I added a macro \subsubparagraph, which does not exist in aaai style and does not violate the camera-ready submission rule. This macro adds a horizontal line and a small note in the marginpar. The code is in asai.tex, and it should be commented out finally (and it becomes a \relax ).

Other overall implementation strategy should be added here. When some new principle is introduced, it should be here.
 
1. Not to say ``CAP is designed for decomposable problems''. It gives an wrong impression that it cannot solve nondecomposable problem. Instead, ``CAP is able to solve large problems, detecting the decomposable parts of the problem in an automated manner. If the problem is not decomposable, then it's ability to solve it solely depends on the underlying MainPlanner. If MainPlanner X can or cannot solve it, then CAP(X) can or cannot solve it, respectively.''

2. too much ``''s make the sentence look scattered and visually less recognizable. ``e.g.'' also.

3. \em, \bf, \it are all obsolete \TeX primitives, and it does not take effect properly --- for example, {\bf {\it aaa}} shows ``aaa'' in italic but NOT IN BOLD. Use \emph{}, \textit{}, \textbf{} and so on.

4. always use \ff, \fd, \cea, \pr, \mv , and do not use it directly, e.g. FF, FD/LAMA2011, etc. 

5. use of footnotes should be minimized.

6. No more mention on R2 or R1 of marvin, because it just cause a confusion.

7. IPC2011 should always be \ipc . The definition can later be modified in abbrev.sty .

8. assembly instances are 20.

9. macro actions -> operators.
10. always use \domind and never ``domain-independent'' nor ``domain independent''.
11. ``planner independent'', not ``planner-independent''.
12. "best first", not "best-first"
13. "mass manufacturing", not "mass-manufacturing"
14. "Component Abstraction", not "component-abstraction"
15. Table, Figure, Fig., are not allowed. Always use \refig and \reftbl. When the development flag is enabled, direct use of \ref signals an error.
16. Caption ends with a period. 
\end{hidden}

%%%%%%%%%%%%%%%%%%%%%%%%%%%%%%%%%%%%%%%%%%%%%%%%%%%%%%%%%%%%%%%%

\begin{abstract}
Maintaining the diversity in the results of heuristic search has been an
important requirement in the applications of automated planning such as
Penetrating Testing.
Despite the common-sense notion of diversity that it provides a new
perspective and better understanding of the world, little has been investigated on the
benefit of diversity when applied to the admissible search algorithms in terms of search efficiency.
In this paper, we propose a novel 
diversified tie-breaking method for A* 
which tries to expand the dissimilar branches as much as possible,
maintaining the admissibility of the search.
\end{abstract}

[Paper Length Limit: 6+1 pages in AAAI, 8+1 pages in ICAPS]

In the applications of automated planning such as Penetrating Testing,
maintaining the diversity of the attack plans is an important requirement from the customers because
such diversity provides a broader insight on
the weakness of the corporation network against 
the recent sophisticated malicious attacks (Hoffman 2004).
To meet these demands, several diversity metrics were developped (Nguyen
12, Roberts et al. 14, Goldman 15) along with several algorithms that
returns a set of plans.

Diversity, or more frequently referred to as the problem of \emph{exploration} versus
\emph{exploitation}, was also considered beneficial in improving the search efficiency.
\emph{k-Best First Search} (Felner, Kraus
and Korf 2003) and \emph{Diverse Best First Search(DBFS)}
(Imai and Kishimoto 11) were developped for
improving the search behavior of Greedy Best First Search by shifting the
balance slightly from exploitation to exploration, because GBFS
is sometimes misdirected by the errors in the heuristic estimates,
and stays too narrowly focused on this wrong direction.

In Robotics, RRT(Rapidly-exploring Random Tree) algorithm is a randomized
algorithm for continuous space and is
widely known for its provabliy good balance on exploration and exploitation.
In this light, several work (Alcazar and
Veloso 11, Burfoot et al. 06, Likhachev 08) have tried to merge
RRT-based algorithms to classical, deterministic planning in discrete state space.
They wrapped a guided search with randomized tree construction which allows the
planner to search the state space scarcely, and its scarceness has a
significant mathematecally good characteristics such as uniform
distribution and convergence.


Despite these interests, the role of diversity in \(A^*\)-based search
algorithm is underinvestigated.
DBFS is a specific improvement to GBFS, and is theoretically less insightful compared
to bare \(A^*\), which is simpler and can be admissible.
RRT-based planners do not significantly improve upon the current
state-of-the-art planner such as LAMA (Alcazar and Veloso 11), although it
has improved in some domains. RRT-based planners are also subjective to
the probabilistic convergence, which means the optimality is guaranteed
only when an infinite amount of search time is provided, or they simply succumb to
inadmissible search.


The contribution of this paper is to add a new variety of improvements to the admissible search.
Recent work by Helmert and Roger (2008) claims that \(A^*\) fails even under an
assumption that an \emph{almost-perfect heuristic} exists, which is an
optimistic and theoretical heuristic function which has only a constant error
behind the perfect heuristics.
They conclude that the further performance improvement requires the
techniques orthogonal to the heuristic function, such as symmetry breaking,
domain reduction, factored planning etc.
Our diversity-based technique falls into this category, possibly
overwrapping with some of them.

The structure of this paper follows: The next section describes the background and
preliminaries on the theoretical aspects of \(A^*\). Next, we describe the
details of our diversified tie-breaking and its theoretical implications.
Then we provide an empirical result of our algorithm compared to
traditional \(A^*\) with various heuristics. Next we show several related
works. We finally conclude with a discussion on the future work.

\section{Backgrounds and Preliminary}
\label{sec-1}

\(A^*\) search and admissibility

almost-perfect heuristic function (depends on the experiments)

RRT, volonoi characteristics, similar states

RRT-based planners

In a continuous space,
if two nodes a, b are very close to each other (the distance is bounded within an epsilon)
and if a, b shares a common parent c,
\textbf{the two paths a-c and b-c are also very similar}.
In this case, adding a new edge b-c is
not beneficial because it is searching the \textbf{almost} same state/path twice.

[ Figure ]

In RRT, however, the search constructs a Rapidly-exploring Random Tree in a continuous space.
Thanks to the randomized point selection and connection strategy in RRT,
we can provably minimize the possibility of adding such almost-duplicated edges.
This is the fundamental motivation of searching the space uniformly in RRT.
In other words, RRT selects a point randomly in the search space and
try to maximize the diversity of the search.

Same thing can be applied in classical planning, especially when we no
longer benefit from the heuristic estimates and have no clue searching the
state space.
This applies not only to the bogus heuristic function h=0, but also to a idealistic
function called almost perfect heuristics h=h*-c. With almost perfect
heuristics, it is reasonable to assume that we no longer improve the
estimates -- it means we have a perfect heuristics.

When we have a large plateau in an admissible search,
we want to diversify the search in the plateau as much as possible.
Within this plateau, several states are expected to be, in a sense,
"similar" or "close" to each other. And the "similar" states are likely to
share the "similar" property regarding the relationship to the goal:
they are either all very close to the goal, or very far from the goal (which means
the heuristic error is large). However, due to the setting we set
above, we cannot improve the estimates any more. Moreover, we are assuming
that we got "nearly all information" toward the goal and there is not much
left in the goal-directed information.
Instead, what was left unexploited is the information "from the initial state to the
current state", which is the target of our contribution.

In A*, the states with the best f value is expanded according to the
tie-breaking criteria. Our contribution is a new tie-breaking tequnique
which diversify the order of expantion within the same f-value.
This tie-breaking is based on already-known information,
without using any goal-directed information.
These information cannot be obtained from the admissible heuristic function even if we
have an almost-perfect heuristics.
This is a kind of knowledge-free exploration similar to RRT.
We call this new algorithm \texttt{*A*} (star-A-star), which optimizes the node
expansion ordering within the best f, exploiting the goal-free information.
When the expansion order is fifo, then this is equivalent to a normal A*.
\section{Method Overview}
\label{sec-2}

[just a series of ideas, no real practice. The text is a sketch.]

In \texttt{*A*}, we first create a map of the plateau based on the path similarity
distance. This map can have an implicit or explicit representation.
Using this map, we diversify the order of expanding each node.

In RRT the map is never created explicitly.
In fact, when there is n nodes, creating such a map requires n(n-1) computation of the distance
between them and this could be expensive. Instead, RRT randomly selects a state
and find the nearest node in the tree. Nearest neighbor search is O(N), and
approximated NNS is O(log N).

\subsection{Distance function}
\label{sec-2-1}

The selection of path similarity distance is very important.
In previous attempts to adopt RRT in classical planning, such as
RRT-Plan or RPT, they use Plan Distances between the leaf states as the
metric. This is not so useful due to many reasons:

\begin{itemize}
\item It requires a computationally expensive explicit search each time.
\item Computing the relaxed plan distance is still expensive.
\item Plan distance does not reflect how similar these states are.
\begin{itemize}
\item Example 1: Symmetry: Although symmetric states are naturally very
"similar", their plan distance may be very large, or even
infinite. -- Suppose we are standing at the center of a straight
road which is 2km long. We can go 1km to the left or 1km to the
right. There are two goals in both ends. Although the two goals and the
paths are symmetric, the goal distance is 2km long, which is a largest
diameter of this search space.
\item Example 2: Partial Order: Suppose we are in a 3x2 grid where we can go
up or right only. Initial/Goal State is (0,0)/(2,1). Suppose the search
frontiers have the paths (0,0)-(0,1)-(1-1) and (0,0)-(1,0)-(2-0).
They both have a parent node
(1,0) and the paths before this meeting point are the same in terms of
partial order. These are not symmetric. Although these frontiers both reach
the same goal state in 1 step, and their prefix are the same, their
plan distance is infinite since we cannot go down or left.
\end{itemize}
\end{itemize}

\begin{verbatim}
g----------s----------g
\end{verbatim}

\begin{verbatim}
+--o--g
|  |  |
s--+--o
\end{verbatim}

We have several possibilities for computing the similarity distance. Here
are some ideas:

\begin{enumerate}
\item We count the difference of the plan according to the D\(_{\text{stability}}\) in
Diversity Planning(Nguyen 12). However, we also want to take the following elements
into account:
\item the length of the shared prefix of the paths s0-s1 and s0-s2.
\begin{itemize}
\item This is fast but not informative, and exploits no information in
Symmetry and Partial Ordering.
\item This only provides the upper bound of the plan similarity because,
although the path s0-s1 is optimal in the admissible search, there
may be several such optimal paths, and this path prefix distance
does not take those multiple paths into consideration.
\end{itemize}
\item Convert the path into the partial order plan and compute the graph
similarity. I suppose Graph Similarity is computationally hard problem
but is in practice very fast. This would make the similarity
more accurate.
\item Path symmetry. This would also make the similarity more accurate.
\end{enumerate}

Note: these do not require those paths to be completely symmetric or
completely isomorphic partial order path. The benefit is that we can delay
the expansion of "mostly" symmetric path/isomorphic partial order path,
although we may not be able to prune them completely.

Note: The role is similar to globally admissible heuristics, which is in turn
found to be a "path-dependent heuristics". However, these information are
highly likely to be considered by an ideal "almost perfect" heuristics.
Those ideal heuristics take into consideration \emph{any possible} information
toward the goal, but \emph{not} about the already-known information.

Note: Another problem in previous RRT-based planners is how to query a reacheable
state randomly, uniformly and efficiently. They run "generate, test and
dispose" approach, however we do not need this because we only apply our
ordering algorithm to the search frontier with best-f. Open-nodes are
always reached from the initial states.

Now the problem is how to diversify the selection from the best-f states,
without creating a distance table which runs n(n-1)/2 iteration.
We instead create the table in an incremental fasion during the search.
We expect a new f-value is obtained before every node is expanded.
Since it does not modify the expansion order regarding f, the algorithm
does not harm the admissibility.

Consider we have a set of nodes n1-n5. It has the following
pairwise distances, which is actually unknown prior to the search.

\begin{center}
\begin{tabular}{r|rrrrr|}
 &  &  &  &  & \\
 & 1 & 2 & 3 & 4 & 5\\
\hline
1 & $\backslash$ & 1 & 7 & 2 & 9\\
2 & 1 & $\backslash$ & 3 & 8 & 5\\
3 & 7 & 3 & $\backslash$ & 4 & 6\\
4 & 2 & 8 & 4 & $\backslash$ & 0\\
5 & 9 & 5 & 6 & 0 & $\backslash$\\
\hline
\end{tabular}
\end{center}

The first node is selected arbitrarly. Let's assume it is n1.
The second node is the farthest node from the root node, which is n5. It
takes 4 distance computation.
Now which node should be the third?
We want to select a node which is diverse from both nodes.
From the table, we estimate n3 is far from both node.
When multiple nodes are accounted, we use a harmonic mean of the distances.
It implies that if some state returns a distance 0 (e.g. they are
symmetric), it is considered only in the last. (of course, it is posible to
prune these states\ldots{})
\subsection{talk with alexf}
\label{sec-2-2}

is path distance really needed? (maybe just a state similarity is enough)
compute plan similarity with zobrist-hash like fast method?

\section{Results}
\label{sec-3}

{ \setlength{\tabcolsep}{0.1em}
\begin{table}[htb]
\caption{The coverage results of \(A^*\) and \(^*A^*\) with 30 min experiments on 2 GB machine.}
\centering
\begin{tabular}{l|ll|ll|ll|ll|}
\footnotesize & blind &  & LMcut &  & Almost Perfect &  & Perfect & \\
 &  &  &  &  & \(c=3\) &  & (\(c=0\)) & \\
Domains & \(A^*\) & \(^*A^*\) & \(A^*\) & \(^*A^*\) & \(A^*\) & \(^*A^*\) & \(A^*\) & \(^*A^*\)\\
\hline
Gripper(20) & 9 & \textbf{13} &  &  &  &  &  & \\
Depot(20) & \textbf{11} & \textbf{11} &  &  &  &  &  & \\
Airport(30) & \ldots{} &  &  &  &  &  &  & \\
TPP &  &  &  &  &  &  &  & \\
CyberSec &  &  &  &  &  &  &  & \\
Sokoban &  &  &  &  &  &  &  & \\
\ldots{} &  &  &  &  &  &  &  & \\
\ldots{} &  &  &  &  &  &  &  & \\
\ldots{} &  &  &  &  &  &  &  & \\
\ldots{} &  &  &  &  &  &  &  & \\
\ldots{} &  &  &  &  &  &  &  & \\
\ldots{} &  &  &  &  &  &  &  & \\
\ldots{} &  &  &  &  &  &  &  & \\
\ldots{} &  &  &  &  &  &  &  & \\
\ldots{} &  &  &  &  &  &  &  & \\
\ldots{} &  &  &  &  &  &  &  & \\
\ldots{} &  &  &  &  &  &  &  & \\
\ldots{} &  &  &  &  &  &  &  & \\
\ldots{} &  &  &  &  &  &  &  & \\
\ldots{} &  &  &  &  &  &  &  & \\
\ldots{} &  &  &  &  &  &  &  & \\
\ldots{} &  &  &  &  &  &  &  & \\
\ldots{} &  &  &  &  &  &  &  & \\
\ldots{} &  &  &  &  &  &  &  & \\
\ldots{} &  &  &  &  &  &  &  & \\
\ldots{} &  &  &  &  &  &  &  & \\
\ldots{} &  &  &  &  &  &  &  & \\
\ldots{} &  &  &  &  &  &  &  & \\
\ldots{} &  &  &  &  &  &  &  & \\
\ldots{} &  &  &  &  &  &  &  & \\
\ldots{} &  &  &  &  &  &  &  & \\
\ldots{} &  &  &  &  &  &  &  & \\
\ldots{} &  &  &  &  &  &  &  & \\
\ldots{} &  &  &  &  &  &  &  & \\
\ldots{} &  &  &  &  &  &  &  & \\
\ldots{} &  &  &  &  &  &  &  & \\
\ldots{} &  &  &  &  &  &  &  & \\
\ldots{} &  &  &  &  &  &  &  & \\
\ldots{} &  &  &  &  &  &  &  & \\
\ldots{} &  &  &  &  &  &  &  & \\
Zenotravel &  &  &  &  &  &  &  & \\
\hline
Total & 800 & \textbf{900} &  &  &  &  &  & \\
\end{tabular}
\end{table}

}
\section{Related Work}
\label{sec-4}

globally-admissible heuristics

symmetry breaking

HTN

factored planning, CAP
\section{Discussion and Conclusion}
\label{sec-5}
